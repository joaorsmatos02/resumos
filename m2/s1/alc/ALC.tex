\documentclass[10pt,a4paper]{report}
\usepackage[utf8]{inputenc}
\usepackage[english]{babel}
\usepackage[T1]{fontenc}
\usepackage{amsmath}
\usepackage{amsfonts}
\usepackage{graphicx}
\usepackage{lmodern}
\usepackage{amssymb}
\usepackage{verbatim}
\usepackage{float}
\usepackage{amsthm}
\usepackage{hyperref}
\title{\LARGE{Algorithms for Computational Logic} \\ \vspace{0.5cm} \normalsize{Summary}}
\date{}

\begin{document}
\maketitle
\tableofcontents

\chapter{SAT and Modeling with SAT}
\section{Cardinality Constraints}
In order to handle cardinality constraints we have two options: encode the cardinality constraints to CNF and use a SAT solver, or use a pseudo boolean (PB) solver.
\subsection{AtMost1}
\begin{itemize}
\item $\sum_{j=1}^n x_j = 1$ can be encoded with $\left(\sum_{j=1}^n x_j \leq 1\right) \land \left(\sum_{j=1}^n x_j \geq 1\right)$
\item $\sum_{j=1}^n x_j \geq 1$ can be encoded with $(x_1 \lor x_2 \lor ... \lor x_n)$
\item $\sum_{j=1}^n x_j \leq 1$ can be encoded with:
\begin{itemize}
\item Pairwise encoding
\item Sequential counter encoding
\item Bitwise encoding
\end{itemize}
\end{itemize}
\subsubsection{Sequential Counter}
In order to realize this encoding, we need to add new variables $s_i$ for the fact "there is a 1 on some position $1..i$":
$$
s_i \text{ is true if } \sum_{j=1}^i x_j \geq 1
$$
Encoding $\sum_{j=1}^n x_j \leq 1$ with sequential counter:
\begin{align*}
&(\neg x_1 \lor s_1) \land\\
&(\neg x_i \lor s_i), i \in 2..n-1 \land\\
&(\neg s_{i-1} \lor s_i), i \in 2..n-1 \land\\
&(\neg x_i \lor \neg s_{i-1}), i \in 2..n
\end{align*}
If $x_j = 1$, then all $s_i$ variables are assigned and all other $x$ variables must take value 0. There are $\mathcal{O}(n)$ clauses and $\mathcal{O}(n)$ auxiliary variables.
\subsubsection{Bitwise Encoding}
In bitwise encoding, we represent the constraint $\sum_{j=1}^n x_j \leq 1$ by encoding the index of the potential true variable in binary. For this, we add new auxiliary variables:
$$
v_0, ... v_r-1;\; r=\lceil \log n \rceil (\text{with } n > 1)
$$
Each variable $x_j$ is assigned a unique binary number that represents its index. Then, for each variable $x_j$ with binary index representation $i$, we create clauses that enforce the condition: 
\begin{itemize}
    \item If $x_j = 1$, assignment to $v_i$ variables must encode $j - 1$, and all other $x$ variables must take value 0
    \item If all $x_j = 0$, any assignment to $v_i$ variables is consistent
\end{itemize}
For example, $x_1 + x_2 + x_3 \leq 1$:\\
\begin{minipage}{0.4\textwidth} 
    \centering
    \begin{table}[H]
        \begin{tabular}{ccc}
              & $j-1$ & $v_1v_0$ \\ \cline{1-3}
        $x_1$ & 0     & 00   \\
        $x_2$ & 1     & 01  \\
        $x_3$ & 2     & 10   
        \end{tabular}
    \end{table}
\end{minipage}
\begin{minipage}{0.55\textwidth} 
    \begin{align*}
        &(\neg x_1 \lor \neg v_1) \land (\neg x_1 \lor \neg v_0)\\
        &(\neg x_2 \lor \neg v_1) \land (\neg x_2 \lor v_0)\\
        &(\neg x_3 \lor v_1) \land (\neg x_3 \lor \neg v_0)\\
    \end{align*}
\end{minipage}
There are $\mathcal{O}(n\log n)$ clauses and $\mathcal{O}(\log n)$ auxiliary variables

\subsection{General Cardinality Constraints}
Constraints of the form $\sum_{j=1}^n x_j \leq k$ or $\sum_{j=1}^n x_j \geq k$ can be added with:
\begin{itemize}
    \item Sequential Counters
    \item BDDs
    \item Sorting Networks
    \item Cardinality Networks
    \item Totalizer
\end{itemize}
\subsubsection{Sequential Counter Encoding}
For each variable $x_i$, create k additional variables $s_{i,j}$ that are used as counters:
\begin{itemize}
    \item $s_{i,j} = 1$ if at least $j$ variables $\{x_1 ... x_i\}$ are assigned value 1
    \item $s_{i,j} = 0$ if at most $j - 1$ variables $\{x_1 ... x_i\}$ are assigned value 1
\end{itemize}
Encoding:
\begin{figure}[H]
    \centering
    \includegraphics[scale=0.4]{1.png}
\end{figure}
\subsubsection{Totalizer Encoding}
In this encoding we count in unary how many of the $n$ variables $(x_1 ... x_n)$ are assigned to 1. It can be visualized as a tree:
\begin{itemize}
    \item Each node is ($name$ : $variable$ : $sum$)
    \item Root node has the output variables $(o_1 ... o_n)$ that count how many variables are assigned to 1
    \item Literals are at the leaves
    \item Each node counts in unary how many leaves are assigned to 1 in its subtree
    \item Example: if $b_2 = 1$, then at least 2 of the leaves $(x_3, x_4, x_5)$ are
    assigned to 1
\end{itemize}
\begin{figure}[H]
    \centering
    \includegraphics[scale=0.5]{2.png}
\end{figure}
To encode $x_1 + x_2 + x_3 + x_4 + x_5 \leq 3$ just set $o_4=0$ and $o_5=0$.\\
Encoding:
\begin{figure}[H]
    \centering
    \includegraphics[scale=0.5]{3.png}
\end{figure}
There are $\mathcal{O}(n\log n)$ new variables and $\mathcal{O}(n^2)$ new clauses

\chapter{Optimization problems and SAT-Based Problem Solving}
\chapter{Satisfiability Modulo Theories}
\chapter{Answer Set Programming}
\end{document}
