\documentclass[10pt,a4paper]{report}
\usepackage[utf8]{inputenc}
\usepackage[portuguese]{babel}
\usepackage[T1]{fontenc}
\usepackage{amsmath}
\usepackage{amsfonts}
\usepackage{graphicx}
\usepackage{lmodern}
\usepackage{amssymb}
\usepackage{verbatim}
\usepackage{float}
\usepackage{minitoc}
\usepackage{hyperref}
\title{\LARGE{Matemática Discreta} \\ \vspace{0.5cm} \normalsize{Resumo}}
\date{}
\renewcommand{\mtctitle}{Conteúdos}

\begin{document}
\maketitle
\tableofcontents

\chapter{Teoria dos Números}
\section{Conjuntos e Funções}
\subsection{Conjunto Potência}
O conjunto potência de $X$ (ou conjunto das partes de $X$) é dado por:
$$
\mathcal{P}(X) = \{A:A\subset X\}
$$
Tem-se que $|\mathcal{P}(X)| = 2^{|X|}$\\
Exemplo: se $X = \{0,4,\alpha\}$, então:
$$
\mathcal{P}(x) = \{\emptyset,\{0\}, \{4\}, \{\alpha\}, \{0,4\}, \{0,\alpha\}, \{4,\alpha\}, X\}
$$
\subsection{Funções}
Uma função $f: X \rightarrow Y$ é uma associação em que a cada $x \in X$ corresponde um único $y \in Y$ tal que $y = f(x)$ em que $X$ é o conjunto de partida e $Y$ é o conjunto de chegada.\\
\\
A imagem de $f$ é $f(X) = \{f(x)\in Y:x\in X\}\subset Y$, logo a imagem de $A \subset X$ é $f(A) = \{f(x)\in Y:x\in A\}\subset Y$. A imagem inversa de $B \subset Y$ é $f^{-1}(B) = \{x\in X:f(x)\in B\}\subset X$
\subsubsection{Classificação}
Dada uma função $f: X\rightarrow Y$:
\begin{itemize}
\item $f$ é injetiva se $\forall x, y \in X, x \neq y \Rightarrow f(x)\neq f(y)$
\item $f$ é sobrejetiva se $f(X) = f(Y)$
\item $f$ é bijetiva se é injetiva e sobrejetiva
\end{itemize}
Se $|X| > |Y|$ então $f$ não pode ser injetiva e se $|X| < |Y|$ não pode ser sobrejetiva.
\subsection{Relações de Equivalência}
Relação de equivalência em $X$ é uma partição de $X$ em subconjuntos disjuntos em que cada subconjunto fica com os objetos equivalentes.\\
\\
Exemplo: Se $X = \{x,y,\zeta,\phi\}$ podemos tornar equivalentes as letras do mesmo alfabeto, obtendo a relação $\{\{x,y\}, \{\zeta,\phi\}\}$.\\
\\
Uma relação $\sim$ é uma relação de equivalência em $X$ se, e só se, para todos os $x, y, z \in X$ é:
\begin{itemize}
\item Reflexiva: $x \sim x$
\item Simétrica: se $x \sim y$ então $y \sim x$
\item Transitiva: se $x \sim y$ e $y \sim z$ então $x \sim z$
\end{itemize}
\section{Teorema Fundamental da Aritmética}
\subsection{Divisibilidade}
Sejam $a, b \in \mathbb{N}$, com $a > b > 1$. A divisão inteira de $a$ por $b$ é a representação:
$$
a = q \cdot b + r
$$
onde $r \in \{0, 1, 2, ..., b-1\}$ é o resto e $q$ o quociente, únicos.\\
\\
Para $a, b \in \mathbb{Z}$ dizemos que $b$ $divide$ $a$ (ou é $divisor$) e escreve-se $b \mid a$ (caso contrário $b \nmid a$) se:
\begin{itemize}
\item $a$ é múltiplo de $b$
\item $\frac{a}{b} \in \mathbb{Z}$
\item o resto da divisão de $a$ por $b$ é zero
\end{itemize}
$$
Div(a) = \{n \in \mathbb{N} : n \mid a \}
$$
\subsubsection{Propriedades}
\begin{itemize}
\item $0 \nmid a$, $1 \mid a, \forall a \in \mathbb{N}$
\item Se $a \mid b$ e $b \mid c$ então $a \mid c$
\item Se $b \mid a$ e $a \mid b$ então $|a| = |b|$
\item Se $a, b \in \mathbb{N}$ e $a \mid b$, então $a \leq b$ e $a \in Div(b)$
\item Se $a \mid b$ então $a \mid bc$
\item Se $a \mid b$ e $a \mid c$ então $a \mid (b+c)$ e $a \mid (b-c)$
\end{itemize}
Um número $p \in \mathbb{N}$ é primo se $Div(p) = \{1,p\}$
\subsection{MDC e Algoritmo de Euclides}
O máximo divisor comum entre $a, b \in \mathbb{N}$ é o maior elemento $(a,b) = mdc(a,b)$ do conjunto:
$$
Div(a) \cap Div(b)
$$
Para determinar $(a,b)$, fazemos uma sucessão de divisões inteiras, começando com $a = d_0, b = d_1$ (supondo $a > b$):
\begin{align}
d_0 &= q_1d_1 + d_2\\
d_1 &= q_2d_2 + d_3\\
&\vdots\\
d_{k-2} &= d_{k-1}q_{k-1} + d_k\\
d_{k-1} &= d_kq_k + 0
\end{align}
No final, obtém-se $(a,b) = d_k$\\
\\
Diz-se que $a,b \in \mathbb{N}$ são primos entre si se $(a,b) = 1$
\subsection{Equação de Bézout}
Sejam $a, b \in \mathbb{N}$ e $d = (a, b)$. A equação de Bézout é:
$$
ax + by = d
$$
Uma solução particular obtém-se do algoritmo de Euclides estendido. Geralmente, temos a equação diofantina:
$$
ax + by = c
$$
Esta equação tem solução $(x_0, y_0)$ se e só se $d \mid c$ e, quando existem, as soluções são infinitas. A solução geral é:
$$
x = x_0 + k\frac{b}{d}, \hspace{0.5cm} y = y_0 - k\frac{a}{d}, \hspace{0.5cm} k \in \mathbb{Z}
$$
Sendo $x_0, y_0$ solução de $ax + by = c = md$.
\subsubsection{Exemplo do Algoritmo de Euclides Estendido}
Para determinar todas as soluções de $711x + 132y = 6$, constrói-se a seguinte tabela, em que $q_i$ é obtido ao aplicar o algoritmo de euclides:
\begin{table}[H]
\centering
\begin{tabular}{llll}
 $d_1$&$-q_i$& $x_i$& $y_i$ \\ \hline
711 &    & 1  & 0   \\
132 & -5 & 0  & 1   \\
51  & -2 & 1  & -5  \\
30  & -1 & -2 & 11  \\
21  & -1 & 3  & -16 \\
9   & -2 & -5 & 27  \\
3   &    & 13 & -70
\end{tabular}
\end{table}
Logo, como $(711,132) = 3$, sabe-se que as soluções de $711x + 132y = 3$ são $x = 13$ e $y = -70$. Dado que $6 = 2 \cdot 3$, tem-se que $711(2\cdot 13) + 132(2 \cdot (-70)) = 6$ e a solução geral é:
$$
x = x_0 - 44k, \hspace{0.5cm} y = y_0 + 237k, \hspace{0.5cm} k \in \mathbb{Z}.
$$
\subsection{Enunciado do TFA}
Seja $n \in \mathbb{N}$.\\
\\
Versão 1: Existe fatorização:
$$
n = p_1...p_m, \hspace{1cm} p_1, ..., p_m \hspace{0.2cm} são \hspace{0.2cm} primos
$$
Versão 2: Existe fatorização:
$$
n = p_1^{e_1}...p_k^{e_k} \hspace{1cm} p_1, ..., p_k \hspace{0.2cm} são \hspace{0.2cm} primos \hspace{0.2cm} distintos, e_j \in \mathbb{N}
$$
Ambas as fatorizações são únicas, a menos de reordenação dos fatores.\\
\\
Corolário:\\
\\
Seja $n = p_1^{e_1}...p_k^{e_k} \in \mathbb{N}$. O conjunto dos divisores positivos de $n$ é:
$$
Div(n) = \{p_1^{c_1}...p_k^{c_k} : c_i \in \{0, ..., e_i\}\}
$$
\section{Congruências}
Dado $m \in \mathbb{N} \geq 2, a, b \in \mathbb{Z}$, dizemos que $a$ é congruente com $b$ módulo $m$
$$
a \equiv b \mod m
$$
se $m \hspace{0.1cm}| \hspace{0.1cm} (b - a)$, isto é, $\exists x \in \mathbb{Z} : a = m \cdot x + b$, ou seja, $m$ divide $a$ com resto $b$.
\subsection{Invertibilidade}
$a \in \mathbb{Z}$ é invertível$\mod m$ ($\exists x \in \mathbb{Z}$ com $ax \equiv 1 \mod m$) se, e só se $(a,m) = 1$.\\
%\\
%No caso de $(a,m) = 1$, tem-se que $ax \equiv ac \mod m$, logo $x \equiv c \mod m$
\subsection{Equações Lineares}
A equação linear modular, de módulo $m \geq 2$ é da forma:
$$
ax \equiv b \mod m
$$
com $a,b \in \mathbb{Z}$. Seja $d = (a,m)$:
\begin{itemize}
\item Não há soluções se $d \nmid b$
\item Se $d = 1$ há uma solução: $x_0 \equiv a^{-1}b \mod m$
\item Se $d \neq 1$ (e $d \mid b$), resolve-se a equação reduzida (dividir por $d$):
$$
a'x \equiv b' \mod m'
$$
em que $a' = \frac{a}{d}, b' = \frac{b}{d}, m' = \frac{m}{d}$. A solução geral é:
$$
x = x_0 + km' \mod m
$$
com $k \in [d]_0$
\end{itemize}
\subsubsection{Teorema Chinês dos Restos}
O Teorema Chinês dos Restos permite resolver sistemas lineares modulares da forma:\\
\begin{equation}
\begin{cases}
x \equiv b_1 \mod m_1\\
x \equiv b_2 \mod m_2\\
...\\
x \equiv b_r \mod m_r
\end{cases}
\end{equation}
Se $(m_i, m_j) = 1, \forall i \neq j$, então o sistema tem uma única solução $\mod M = m_1m_2...m_r$:
$$
x \equiv b_1 \frac{M}{m_1}y_1 + ... + b_r \frac{M}{m_r}y_r \mod M
$$
em que $y_k = \left(\frac{M}{m_k}\right)^{-1} \mod m_k, k = 1, ..., r$.
\subsubsection{Pequeno Teorema de Fermat}
Se $p$ é primo, e $a$ não é múltiplo de $p$, então:$$
a^{p-1} \equiv 1 \mod p
$$
\subsection{Função Totiente de Euler}
A função totiente é definida por:
$$
\varphi(n) = \mid\{x \in [n]_0 : (x, n) = 1\}\mid
$$
Trata-se da cardinalidade do conjunto dos números entre 0 e $n - 1$ que são primos com $n$ (número de invertíveis em $\mathbb{Z}_n$).\\
\\
Se $p$ é primo, então $\varphi(p) = p-1$. Mais geralmente, $\varphi(p^r) = p^r - p^{r-1}, r \geq 1$. Se $(n,m) = 1$ temos $\varphi(nm) = \varphi(n)\varphi(m)$.\\
\\
Corolário: Sendo $n = p_1^{k_1}...p_r^{k_r}$, temos:
$$
\varphi(n) = n \left(1-\frac{1}{p_1}\right)...\left(1-\frac{1}{p_r}\right)
$$
\subsubsection{Teorema de Euler}
Euler generalizou o pequeno teorema de Fermat para qualquer módulo:\\
\\
Se $(a,n) = 1$, então:
$$
a^{\varphi(n)} \equiv 1 \mod n
$$
\subsubsection{Teorema de Daniel Augusto da Silva}
Se $n_1, ..., n_r$ são primos entre si, e $n = n_1...n_r$, então:
$$
\sum_{i=1}^{r} n_i^{\varphi(n)/\varphi(n_i)} \equiv r - 1 \mod n
$$
\section{Criptografia Clássica}
Seja $M$ a mensagem a enviar e $C$ a mensagem codificada.\\
\subsection{Cifra de César}
Na cifra de césar usam-se números de 0 a 25 (mod 26) para as letras:
$$
C = M + k \mod 26
$$
\subsection{Funções de um só sentido}
Uma função de um só sentido é uma função $f: X \rightarrow Y$ em que $f(x)$ tem uma baixa complexidade computacional $\forall x \in X$, mas $f^{-1}(y)$ possui uma elevada complexidade computacional $\forall y \in Y$.
\subsection{Algoritmo RSA}
Módulo base: São escolhidos $p$ e $q$, primos distintos e calcula-se $N = pq$.\\
Módulo expoente: $\varphi(N) = (p - 1)(q - 1)$\\
\\
Passo 1: O recetor escolhe um número invertível mod $\varphi(N)$, $e$.\\
\\
Passo 2: Calcula $d = e^{-1} \mod \varphi(N)$ usando, por exemplo, o algoritmo de Euclides estendido\\
\\
Passo 3: Publica $(N,e)$, mantendo $d$, $\varphi(N)$, $p$ e $q$ em segredo\\
\\
Passo 4: O emissor calcula $C = M^e \mod N$ e envia $C$ (a mensagem codificada)\\
\\
Passo 5: O recetor calcula $C^d = (M^e)^d$ e descobre que $C^d \equiv M \mod N$ (Teorema de Euler)
\subsubsection{Teorema RSA}
Sejam $p$, $q$ primos, $N = pq$ e $e$, $d$ inversos um do outro mod $\varphi(n)$. Então:
$$
x^{ed} \equiv x \mod N \hspace{0.7cm} \forall x < min\{p,q\}
$$
\subsubsection{Exemplo}
Sejam $p = 61$ e $q = 53$.\\
Então $N = pq = 3233$ e $\varphi(N) = 60 \cdot 52 = 3210$.\\
Escolhe-se $e = 661$ e determina-se $d = e^{-1} \equiv 1501 \mod 3120$.\\
A chave de encriptação $(N,e) =  (3233,661)$.\\
\\
Seja então a mensagem a enviar $M = x = 2762$. Ao ser codificada obtém-se $C = y = 2762^{661} \equiv 78 \mod 3233$.\\
Calcula-se então $78^{1501} \equiv 2762 \mod 3233$, recuperando $M$.

\chapter{Teoria dos Conjuntos}
\section{Números Binomiais}
\subsection{Arranjos sem Repetição}
O número de arranjos sem repetição de $n$ elementos $k$ a $k$ em x é:
$$
A_k^n = n(n-1) ... (n-k+1) = \frac{n!}{(n-k)!}
$$
\subsection{Combinações}
Com $n \geq m \geq 0$ definimos o número binomial (número de combinações de $n$ elementos $m$ a $m$):
$$
\begin{pmatrix}
n\\
m
\end{pmatrix} = \frac{n!}{m!(n-m)!}
$$
\subsubsection{Propriedade Fundamental}
$$
\begin{pmatrix}
n+1\\
m
\end{pmatrix} = \begin{pmatrix}
n\\
m
\end{pmatrix} + \begin{pmatrix}
n\\
m-1
\end{pmatrix}
$$
\subsection{Binómio de Newton}
Sendo $n \in \mathbb{N}$, temos a seguinte igualdade de polinómios:
$$
(x+y)^n = \sum_{k=0}^{n} \begin{pmatrix}
n\\
k
\end{pmatrix}
x^ky^{n-k}
$$
\subsection{Números Multinomiais}
Sendo $n = n_1 + ... + n_k \in \mathbb{N}$, com $n_1, ..., n_k \geq 1$, definimos:
$$
\begin{pmatrix}
n\\
n_1, ..., n_k
\end{pmatrix} = \frac{n!}{n_1!...n_k!}
$$
\section{Princípios da Combinatória}
\subsection{Princípio da Inclusão-Exclusão}
O PIE determina o cardinal de uma união não disjunta de conjuntos. Para 2 conjuntos:
$$
|A \cup B| = |A| + |B| - |A \cap B|
$$
Para 3 conjuntos:
$$
|A \cup B \cup C| = |A|+|B|+|C| - |A \cap B| - |A \cap C| - |B \cap C| + |A \cap B \cap C|
$$
\subsubsection{Fórmula Geral}
Seja $A_{i_1i_2...i_k} = A_{i_1} \cup A_{i_2} \cup ... \cup A_{i_k}$. Tem-se:
$$
\Bigg | \bigcup_i A_i \Bigg | = \sum_{i=1}^n |A_i| - \sum_{i_1<i_2} |A_{i_1i_2}| + ... + (-1)^{k-1} \sum_{i_1 < ... < i_k} |A_{i_1...i_k}| + ... + (-1)^{n-1} |A_{1...n}|
$$
\subsection{Forma Complementar do PIE}
A forma complementar do PIE calcula o cardinal do complementar de uma união:
$$
| X \setminus (A_1 \cup A_2 \cup ... \cup A_n) | = |A_1^c \cap A_2^c \cap ... \cap A_n^c |
$$
Onde $A_j^c = X \setminus A_j$. De um modo geral:
$$
\Bigg | X \setminus \bigcup_i A_i \Bigg | = \Bigg | \bigcap_i A_i^c \Bigg | = |X| - \sum_{i = 1}^{n} |A_i| + \sum_{i_1 < i_2} |A_{i_1i_2}| - ... + (-1)^k \sum_{i_1 ... i_k} |A_{i_1 ... i_k}| + ... + (-1)^n |A_{1...n}|
$$
\section{Simetria}
\subsection{Permutações}
Uma permutação de $n$ elementos é uma bijecção $\pi : [n] \rightarrow [n]$. Por exemplo:
$$
\pi = \begin{pmatrix}
1 & 2 & 3 & 4 & 5 & 6\\
2 & 3 & 1 & 4 & 6 & 5
\end{pmatrix}
$$
Representa $\pi(1) = 2, ..., \pi(6) = 5$. $S_n$ representa o conjunto de todas as permutações, $|S_n| = n!$.
\subsubsection{Notação cíclica}
$$
\pi = (123)(56) \in S_6
$$
Pois $\pi(1) = 2$, $\pi(2) = 3$, $\pi(3) = 1$ (completando o ciclo) e $\pi(5) = 6$ (esgotando os elementos).
\subsection{Grupos Finitos}
Chama-se grupo finito a um conjunto finito $G$ com operação associativa $\cdot$, elemento neutro $e$, em que todos os elementos têm inverso.\\
\\
Diz-se que $H$ é subgrupo de $G$ ($H \subset G$) se $e \in H$, fechado para composição e inverso.
\subsubsection{Teorema de Lagrange}
Se $H \subset G$ é subgrupo, então:
$$
|G| = |H| |G/H|
$$
Onde $G/H$ é o espaço das classes de equivalência da relação:
$$
x \sim y \hspace{0.2cm} se \hspace{0.2cm} \exists h \in H : y = hx
$$
\subsection{Grupo cíclico $\mathbb{Z}_m$}
A ordem de $g \in G$ é o menor número natural $k$ tal que $g^k = e$.\\
\\
Se $H \subset G$ é subgrupo, então $|H|$ divide $|G|$. A ordem de qualquer elemento de $G$ divide $|G|$.\\
\\
Pode-se pensar em $\mathbb{Z}_m$ como o grupo de rotações de um polígono com $m$ lados.
\subsubsection{Exemplo}
$\mathbb{Z}_8$ corresponde às rotações de um octógono com vértices $\{0,1,...,7\}$
$$
\mathbb{Z}_8 \cong \{e, (01234567), (0246)(1357), (03614725), (04)(15)(26)(37), ...\}
$$
Então $H = \{e,(04)(15)(26)(37)\} \subset \mathbb{Z}_8 = G$ é subgrupo e verifica-se:
$$
8 = |G| = |H||G/H| = 2 \cdot 4
$$
$$
H = \{0,4\} \subset G = \mathbb{Z}_8
$$
Pelo que $G/H$ tem 4 classes.
\subsection{Ação de um Grupo $G$ num conjunto $X$}
Uma ação de $G$ num conjunto $X$ é uma aplicação:
$$
G \times X \rightarrow X, \hspace{1cm} (g,x) \rightarrow g \cdot x \in X
$$
com $e \cdot x = x$ e $(gh) \cdot x = g \cdot (h \cdot x), \forall g, h \in G, x \in X$\\
\\
\\
Órbita de $x$: $G \cdot x = \{g \cdot x : g \in G\} \subset X$\\
\\
Estabilizador de $x$: $G_x = \{h \in H : h \cdot x = x\} \subset G$\\
\\
Conjunto fixo por $g$: $X^g = \{x \in X : g \cdot x = x\} \subset X$
\subsubsection{Teorema da Órbita-Estabilizador}
Para todo $x \in X$:
$$
|G| = |G \cdot x||G_x|
$$
\subsection{Lema de Cauchy-Frobenius-Burnside}
Dada a ação de $G$ em $X$, o espaço das órbitas é $X/G$. O lema Burnside determina o número de órbitas $|X/G|$, a partir dos conjuntos de pontos fixos:
$$
|X/G| = \frac{1}{|G|} \sum_{g \in G} |X^g|
$$
\subsubsection{Colorações de $X$}
Seja $c(g)$ o número de ciclos de $g \in G$, como permutação de $X$. O número de colorações do conjunto $X$, tomando em conta as simetrias dadas por $G$ é:
$$
|X/G| = \frac{1}{|G|} \sum_{g \in G} |X^g| = \frac{1}{|G|} \sum_{g \in G} |X|^{c(g)}
$$
\section{Funções Geradoras e Recorrências}
\subsection{Sucessões e Funções Geradoras}
Seja $(u_n)_{n \in \mathbb{N}_0}  = (u_o, u_1, ..., u_n, ...)$ uma sucessão de números reais. A função geradora associada a $(u_n)$ é a série:
$$
f(x) = \sum_{n = 0}^{\infty} u_nx^n = \sum_{x \geq 0} u_nx^n
$$
Muitas sucessões em combinatória têm funções geradoras racionais:
$$
f(x) = \frac{p(x)}{q(x)} \hspace{1cm} p(x), q(x) \in \mathbb{R}[x]
$$
Sabendo que $p(x)$ é da forma $ax+b$, podemos determinar $a$ e $b$ através das derivadas de $f(x)$, já que $f(0) = u_0$ e $f'(0) = u_1$.
\subsubsection{Exemplos}
\begin{itemize}
\item Série geométrica: $\sum_{n \geq 0} x^n = \frac{1}{1-x}$
\item Newton: $\sum_{n \geq 0} \begin{pmatrix}
N\\
n
\end{pmatrix} x^n = (1+x)^N$
\item Derivar: $\left(\frac{1}{1-x}\right)' = \sum_{n \geq 0} nx^{n-1} = \frac{1}{(1-x)^2}$
\item Exponencial: $\sum_{n \geq 0} \frac{x^n}{n!} = e^x$
\end{itemize}
\subsection{Recorrência Linear}
Uma equação de recorrência de ordem $k \in \mathbb{N}$ é:
$$
u_{n+k} = F(u_n, u_{n+1}, ..., u_{n+k-1}), n \in \mathbb{N}_0
$$
O problema de recorrência de ordem $k$ tem-se quando $u_0 = c_0, ..., u_{k-1} = c_{k-1}$. Uma equação de recorrência linear de ordem $k$ é:
$$
u_{n+k} = a_{k-1}u_{n+k-1} + ... + a_1u_{n+1} + a_0u_n + b(n), n \in \mathbb{N}_0
$$
Sendo homogénea quando $b(n) = 0,  \forall n \in \mathbb{N}$. O polinómio característico desta equação é:
$$
p(x) = x^k - a_{k-1}x^{k-1} - ... - a_1x - a_0
$$
Independentemente de $b(n)$.\\
\\
Se $\lambda$ é raiz do polinómio $p(x)$, então $u_n = \alpha \lambda^n$ é solução da homogénea para qualquer $\alpha \in \mathbb{R}$. Se $u_n$ e $v_n$ são soluções da homogénea, então $\alpha u_n + \beta v_n$ também.
\subsubsection{Solução geral dos PRL}
Sejam $\lambda_1, \lambda_2, ...$ as raízes com multiplicidades $m_1, m_2, ...$ ($\textstyle \sum_i m_i = k$):
\begin{itemize}
\item Caso homogéneo ($b(n) = 0,  \forall n$):
$$
u_n = \alpha_1^{(0)} \lambda_1^n + \alpha_1^{(1)} n\lambda_1^n + ... + \alpha_1^{(m_i - 1)} n^{m_i - 1} \lambda_1^n + ...
$$
Os coeficientes $\alpha_i^{(j)}$ determinam-se com as condições iniciais.
\item Caso não homogéneo:
$$
u_n = v_n + \alpha_1^{(0)} \lambda_1^n + ... + \alpha_1^{(m_i - 1)} n^{m_i - 1} \lambda_1^n + ...
$$
onde $v_n$ é a solução particular, obtida por substituição
\item Função geradora da solução: $f(x) = \frac{q(x)}{x^kp\left(\frac{1}{k}\right)}$, com $q(x)$ de grau $< k$
\end{itemize}

\chapter{Teoria dos Grafos}
\section{Introdução}
Um grafo pode ser representado na forma $\Gamma = (V,A)$, em que cada vértice $v \in V$ e cada aresta $\alpha \in A$. Podem ser classificados da seguinte forma:
\begin{itemize}
\item Simples: $\alpha = \{v, \omega\}, A \subset \mathcal{P}_2(V)$
\item Multigrafo: arestas diferentes $\alpha_2 \neq \alpha_1 \in A$ podem ligar os mesmos vértices: $\phi : A \rightarrow \mathcal{P}_2(V), \phi(\alpha_1) = \phi(\alpha_2)$
\item Pseudo-grafo: admite lacetes, $\phi : A \rightarrow V \sqcup \mathcal{P}_2(V), \phi(\alpha) = \{v\}$
\item Dirigido: cada aresta está orientada $\psi : A \rightarrow V^2 = V \times V, \psi(\alpha) = (v, \omega)$, $v$ inicio, $\omega$ fim.
\end{itemize}
Os extremos da aresta $\alpha = \{v, \omega\} \in A$ são $v$ e $\omega$; $\alpha$ incide em $v$ e em $\omega$.\\
A valência ou grau de $v \in V$ é:
$$
d_v = |\{ \alpha \in A : v \in \alpha\}|
$$
\subsection{Propriedades}
Seja $\Gamma$ um grafo (ou pseudo-grafo). Tem-se:
$$
\sum_{v \in V} d_v = 2 |A|
$$
Diz-se que ($d_1, d_2, ..., d_n$) é sequência gráfica se existe um grafo simples $\Gamma = (V,A), |V| = n$, em que $d_i$ é o grau do vértice $v_i \in V$.\\
Se $\Gamma$ é um grafo simples com graus $d_1 \leq .. \leq d_n$, então:
\begin{itemize}
\item $\sum_{i = 1}^n d_i$ é par
\item $d_n < n$ (implica cada $d_i < n$ e $\sum_{i=1}^n d_i \leq n(n-1)$
\item ($d_1, ..., d_n, d_{n+1} = \Delta$) é sequência gráfica se e só se ($d_1, ..., d_k, d_{k+1} - 1, ..., d_n - 1$) é sequência gráfica, $n = k + \Delta$
\end{itemize}
\subsection{Passeios e Caminhos}
\begin{itemize}
\item Um passeio é uma sequência alternada de vértices e arestas, tal que o elemento seguinte é adjacente/incidente ao anterior, começa e termina em vértices.
\item Um caminho é um  passeio que não repete vértices nem arestas.
\item Um ciclo é um caminho que começa e termina no mesmo vértice.
\end{itemize}
Um caminho é Hamiltoniano se passa por todos os vértices e um passeio diz-se Euleriano se passa por todas as arestas sem repetição.
\subsection{Características}
\subsubsection{Conexidade}
Um grafo diz-se conexo se para quaisquer dois vértices existe um passeio/caminho entre eles.
\subsubsection{Isomorfismo}
Um isomorfismo entre os grafos $\Gamma = (V,A)$ e $\Gamma' = (V',A')$ é uma bijecção $f : V \rightarrow V'$ com $\{v_i,v_j\} \in A$ se e só se $\{f(v_i),f(v_j)\} \in A'$.
\subsection{Árvores}
Uma árvore é um grafo conexo e sem ciclos. Uma floresta é um grafo sem ciclos, ou seja, uma união disjunta de árvores.\\
\\
Numa árvore há um único caminho entre quaisquer dois vértices. ($d_1, ..., d_n$) é a sequência gráfica de uma árvore se e só se $\sum_{i=1}^n d_i = 2n-2$.
\subsubsection{Árvores Geradoras}
Uma árvore geradora de um grafo simples $\Gamma$ é uma árvore em $\Gamma$ que contém todos os seus vértices.
\section{Grafos planares}
Um grafo é planar se pode ser desenhado num plano sem que as arestas se intersectem. Um grafo planar conexo com $v$ vértices, $a$ arestas e $f$ faces verifica:
$$
v - a + f = 2
$$
Contando com a face exterior. Se um grafo é planar conexo com $v > 2$ vértices e $a$ arestas (e faces de ordem $> 2$) temos:
$$
a \leq 3v - 6
$$
\subsection{Grafo Dual}
Dado um grafo planar $\Gamma$, o grafo dual $\Gamma^\lor$ é obtido trocando os vértices com as faces. Por exemplo:
\begin{figure}[H]
\centering
\includegraphics[scale=0.5]{Sem Título1.png}
\end{figure}
\subsection{Teorema de Kuratowski}
Um subgrafo do grafo simples $\Gamma$ é $\Gamma' \subset \Gamma$ obtido tomando só algumas arestas e os seus vértices incidentes.\\
\\
Uma subdivisão de um grafo $\Gamma$ é um novo grafo $\Gamma'$ obtido adicionando alguns vértices no meio de arestas de $\Gamma$.\\
\\
Pelo Teorema de Kuratowski, $\Gamma$ é um grafo planar se e só se não contém nenhum subgrafo que é uma divisão de $K_5$ ou de $K_{3,3}$
\begin{figure}[H]
\centering
\includegraphics[scale=0.5]{Sem Título2.png}
\end{figure}
\section{Matrizes Associadas}
\subsection{Grafos Simples e Multigrafos}
Seja $\Gamma = (V,A)$ um grafo simples ou multigrafo, em que $V = \{v_1, ..., v_n\}, A = \{\alpha_1, ..., \alpha_m\}$ 
\subsubsection{Matriz de Incidência}
A matriz de incidência é uma matriz $m \times n = |A| \times |V|$ tal que:
\[
M_{ij} =
\begin{cases}
1, \hspace{0.7cm} se \hspace{0.1cm} \alpha_i \hspace{0.1cm} incide \hspace{0.1cm} em \hspace{0.1cm} v_j\\
0, \hspace{0.7cm} se \hspace{0.1cm} \alpha_j \hspace{0.1cm} não \hspace{0.1cm} incide \hspace{0.1cm} em \hspace{0.1cm} v_j, \hspace{1cm} i \in [m], j \in [n]
\end{cases}
\]
\subsubsection{Matriz de Adjacência}
A matriz de adjacência é uma matriz quadrada simétrica $n \times n$:
$$
J_{ij} = k
$$
Se $v_i$ e $v_j$ estão ligados por k arestas (= 0 se $i = j$).
\subsubsection{Matriz de Valência}
A matriz de valência é uma matriz diagonal $n \times n$:
$$
D = diag(d_{v_1}, ..., d_{v_n})
$$
Em que $D_{ij} = 0$ para $i \neq j$.\\
\\
Teorema: Para um multigrafo, $M^tM = J + D$.
\subsubsection{Teorema de Kirchhoff}
Uma matriz laplaciana de $\Gamma$ com $n$ vértices é a matriz quadrada ($n \times n$) $L = D - J$. Para cada $v \in V$, seja $L_v$ o correspondente cofator de $L$. Pelo teorema de Kirchhoff, se $\Gamma$ é conexo então:
\begin{itemize}
\item det$L_v$ = det$L_w$, para quaisquer $v,w \in V$
\item det$L_v$ é o número de árvores geradoras em $\Gamma$
\end{itemize}
\subsubsection{Matriz Estocástica}
Um vetor estocástico é $v = (v_1,v_2,...,v_n)$ com $v_i \geq 0$ e $v_1 + v_2 + ... + v_n = 1$.\\
Uma matriz estocástica $n \times n$ é uma matriz cujas colunas são vetores estocásticos. Se nenhuma entrada for zero, diz-se estocástica positiva.\\
\\
Sejam $A$ e $B$ matrizes estocásticas:
\begin{itemize}
\item $AB$ é estocástica
\item $sA + tB$ é estocástica positiva $\forall s,t > 0$ com $s + t = 1$
\end{itemize}
Seja $M$ uma matriz estocástica positiva, pelo teorema de Perron-Frobenius:
\begin{itemize}
\item 1 é o valor próprio de $M$
\item Há um único valor próprio estocástico $p$ de valor próprio 1
\item $M^nv$ tende para $p$, para qualquer $v$ estocástico
\end{itemize}
$p$ denomina-se o vetor de Perron-Frobenius
\subsection{Grafos Dirigidos}
Seja $\Gamma = (V,A)$ um grafo dirigido, em que $V = \{v_1, ..., v_n\}, A = \{\alpha_1, ..., \alpha_m\}$
\subsubsection{Matriz de Incidência}
A matriz de incidência é uma matriz $m \times n$ tal que:
\[
M_{ij} =
\begin{cases}
1, \hspace{1cm} se \hspace{0.1cm} \alpha_i \hspace{0.1cm} aponta \hspace{0.1cm} para \hspace{0.1cm} v_j\\
-1, \hspace{0.7cm} se \hspace{0.1cm} \alpha_i \hspace{0.1cm} sai \hspace{0.1cm} de \hspace{0.1cm} v_j\\
0, \hspace{1cm} se \hspace{0.1cm} \alpha_j \hspace{0.1cm} não \hspace{0.1cm} incide \hspace{0.1cm} em \hspace{0.1cm} v_j, \hspace{1cm} i \in [m], j \in [n]
\end{cases}
\]
\subsubsection{Matriz de Adjacência/Transferência}
A matriz de adjacência/transferência é uma matriz $T$ quadrada não simétrica $n \times n$ com $T_{ij} = 1$ se há seta de $v_i$ para $v_j$ (linha $j$, coluna $i$).\\
\\
Teorema: $(T^n)_{ji}$ é o número de percursos diferentes entre o vértice $v_i$ e $v_j$ com comprimento igual a $n$
\subsubsection{Matriz Estocástica}
Sendo $\Gamma = (V,A)$ é um grafo dirigido. Utiliza-se na notação dot para listar todas as flechas: $1 \rightarrow \{i,j,...\}, 2 \rightarrow \{k,l,...\}$.\\
\\
A matriz estocástica do grafo dirigido com $n$ vértices é uma matriz $n \times n$ que estipula igual probabilidade de seguir as várias flechas de saída e $1/n$ para os vértices sem saída. Por exemplo, para as flechas $1 \rightarrow\{2,3,4\}; 2 \rightarrow \{3,4\}; 3 \rightarrow 1; 4 \rightarrow \emptyset$ constrói-se a seguinte matriz:
$$
\begin{pmatrix}
0 & 0 & 1 & \frac{1}{4}\\
\frac{1}{3} & 0 & 0 & \frac{1}{4}\\
\frac{1}{3} & \frac{1}{2} & 0 & \frac{1}{4}\\
\frac{1}{3} & \frac{1}{2} & 0 & \frac{1}{4}
\end{pmatrix}
$$
Proposição: A probabilidade de transição do vértice $i$ para o $j$ em $k$ passos é a entrada da coluna $i$ e linha $j$ da matriz $E^k$
\section{Algoritmo PageRank}
Supondo que $\Gamma = (V,A)$ com matriz estocástica $E$ é um grafo dirigido que representa a internet, em que os $N$ vértices são páginas e as flechas são links para outras páginas. O algoritmo PageRank calcula o vetor estocástico cuja entrada $i$ é a probabilidade de após $n$ passos estarmos na página $i$ da seguinte forma:
\begin{itemize}
\item Escolher vetor estocástico inicial, usualmente $v = \frac{1}{N}(1,1,...,1)$
\item Escolher peso $\rho \in ]0,1[$ e calcular $H = (1 - \rho)E + \frac{\rho}{N}1$ (onde 1 é a matriz só com 1's), usualmente $\rho = 0.15$
\item Calcular $H^nv$ para $n$ suficientemente grande
\end{itemize}
\end{document}