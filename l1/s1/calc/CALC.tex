\documentclass[10pt,a4paper]{report}
\usepackage[utf8]{inputenc}
\usepackage[portuguese]{babel}
\usepackage[T1]{fontenc}
\usepackage{amsmath}
\usepackage{amsfonts}
\usepackage{lmodern}
\usepackage{amssymb}
\usepackage{graphicx}
\begin{document}

\chapter{Assintotas}
\section{Horizontais}
Para encontrar as assintotas horizontais de uma função $f(x)$ devemos calcular:
$$
\lim_{x\rightarrow +\infty} f(x)
$$
$$
\lim_{x\rightarrow -\infty} f(x)
$$
Caso seja um qualquer número $n \in \mathbb{R}$, então $n$ é assintota horizontal.
\section{Verticais}
Para encontrar as assintotas verticais de uma função racional $f(x)$ devemos primeiro encontrar os zeros do denominador. De seguida devemos calcular o limite de $f(x)$ quando $x$ tende para esses zeros:
$$
\lim_{x \rightarrow z} f(x)
$$
Caso  seja $+\infty$ ou $-\infty$ então $z$ é assintota vertical de $f(x)$.
\section{Obliquas}
Para encontrar as assintotas obliquas de uma função $f(x)$ devemos primeiro encontrar o seu declive, calculando:
$$
\lim_{x \rightarrow +\infty} \frac{f(x)}{x}
$$
E de seguida encontrar o seu valor de $b$, através de:
$$
\lim_{x \rightarrow +\infty} (f(x)-mx)
$$

\chapter{Limites}
\section{Indeterminações}
As indeterminações mais comuns são:
\begin{itemize}
\item[$\frac{0}{0}$] Fatorizar numerador e denominador
\item[$\frac{\infty}{\infty}$] Dividir o numerador e denominador pelo $x$ de maior grau.
\item[$0 \times \infty$] Formas do tipo $0 \times \infty$ podem ser transformadas em $\frac{0}{\frac{1}{\infty}}$ ou $\frac{\infty}{\frac{1}{0}}$
\end{itemize}
\section{Regra de L'Hopital}
Para resolver indeterminações do tipo $\frac{\infty}{\infty}$ ou $\frac{0}{0}$ é possível utilizar a regra de L'Hopital, que consiste em derivar o numerador e o denominador da fração individualmente e a partir das suas derivadas calcular o limite inicial.

\chapter{Série}

\section{Convergência e divergência}
A série $\sum_{}^{} a_{n}$ diz-se $convergente$ se for limitada. Caso contrário é $divergente$.

\section{Critério de Cauchy-Bolzano}
Uma série $\sum_{}^{} a_n$ é convergente se, e só se dadas quaisquer subsucessões $T_N - U_N$ da sucessão
das somas parciais, a diferença $T_N - U_N$ tende para zero. Logo:
\begin{itemize}
\item Se a série $\sum_{}^{} a_n$ é convergente, então $\lim a_n = 0$
\item Se a série $\sum_{}^{} a_n$ é divergente nada se conclui.
\item Se $\lim a_n \neq 0$ então a série $\sum_{}^{} a_n$ é divergente.
\item Se $\lim a_n = 0$ nada se conclui.
\end{itemize}
Ou seja, toda a série convergente tem limite igual a 0 mas as séries divergentes podem ter qualquer limite.

\section{Série geométrica}
Uma série diz-se geométrica de razão $r$ se é do tipo:
$$
\sum_{}^{} ar^{n}
$$
\subsection{Somas parciais de uma série geométrica}
Considere-se a série geométrica $\sum_{n=0}^{\infty} r^{n}$, com $r \neq 1$. Para cada $N \geq 0$, tem-se:
$$
S_N = \sum_{n=0}^{N} r^{n} = 1 + r + r^2 + ... + r^N = \frac{1-r^{N+1}}{1-r}
$$
\subsection{Convergência da série geométrica}
Uma série geométrica de razão $r$ é convergente se e só se $-1 < r < 1$. Sendo $a_1$ o primeiro termo, a soma da série geométrica convergente é:
$$
a_1 \times \frac{1}{1-r}
$$

\section{Séries redutivas ou de Mengoli}
Uma série diz-se redutiva ou série de Mengoli se é do tipo:
$$
\sum_{n=1}^{\infty} (u_n - u_{n+k})
$$
\subsection{Convergência da série redutiva}
A série $\sum_{n=1}^{\infty} (u_n - u_{n+k})$ é convergente se $u_n$ for convergente. Se $\lim u_n = a \in \mathbb{R}$, a soma da série é:
$$
\sum_{n=1}^{\infty} (u_n - u_{n+k}) = u_1 + ... + u_k - ka
$$

\section{Séries de Dirichlet}
Uma série diz-se de Dirichlet se for da forma $\sum_{}^{} \frac{1}{n^k}$, com $k \in \mathbb{R}$.
\subsection{Convergência da série de Dirichlet}
Uma série de Dirichlet $\sum_{}^{} \frac{1}{n^k}$ converge se, e só se $k > 1$.

\section{Séries de termos positivos}
Uma série $\sum_{}^{} a_n$ diz-se de termos positivos se $a_n > 0$ para todo $n \in \mathbb{N}$. A sucessão de todas as somas parciais de $\sum_{}^{} a_n$ é estritamente crescente.

\section{Resumo}
\begin{table}[h]
\centering
\begin{tabular}{c|c|c}
$\sum$ & Forma & Convergencia\\ \hline
Geométrica & $\sum_{}^{} ar^{n}$ & $-1 < r < 1$\\ \hline
Mengoli & $\sum_{}^{} (u_n - u_{n+k})$ & $u_n$ converge\\ \hline
Dirichlet & $\sum_{}^{} \frac{1}{n^k}$ & $k > 1$\\ \hline
Termos positivos & $a_n > 0, \forall n \in \mathbb{N}$ & -----
\end{tabular}
\end{table}
\begin{itemize}
\item[Geométrica:]
\item[] 
\begin{itemize}
\item[soma total:]
$
a_1 \times \frac{1}{1-r}
$
sendo $a_1$ o primeiro termo.
\item[soma parcial:]
$$
S_N = \sum_{n=0}^{N} r^{n} = 1 + r + r^2 + ... + r^N = \frac{1-r^{N+1}}{1-r}
$$
\end{itemize}
\item[Mengoli:]
\item[] soma total:
$$
\sum_{n=1}^{\infty} (u_n - u_{n+k}) = u_1 + ... + u_k - ka
$$
sendo $a = \lim u_n$.
\end{itemize}

\section{Critérios}
\subsection{Critério da Razão ou de D’Alembert}
Seja $\sum_{}^{} a_n$ uma série de termos positivos e $a = \lim \frac{a_{n+1}}{a_n}$, então:
\begin{itemize}
\item Se $a < 1$ a série é convergente
\item Se $a > 1$ a série é divergente
\item Se $a = 1$ nada se conclui
\end{itemize}
\subsection{Critério da raiz ou de Cauchy}
Seja $\sum_{}^{} a_n$ uma série de termos positivos e $a = \lim \sqrt[n]{a_n}$, então:
\begin{itemize}
\item Se $a < 1$ a série é convergente
\item Se $a > 1$ a série é divergente
\item Se $a = 1$ nada se conclui
\end{itemize}

\section{Critérios de comparação}
\subsection{Primeiro critério de comparação}
Sejam $\sum_{}^{} a_n$ e $\sum_{}^{} b_n$ séries de termos positivos, com $a_n \leq b_n$, pelo menos a partir de certa ordem, então:
\begin{itemize}
\item Se $\sum_{}^{} a_n$ é divergente então $\sum_{}^{} b_n$ é divergente
\item Se $\sum_{}^{} b_n$ é convergente, então $\sum_{}^{} a_n$ é convergente
\end{itemize}
\subsection{Segundo critério de comparação}
Sejam $\sum_{}^{} a_n$ e $\sum_{}^{} b_n$ séries de termos positivos e $L = \lim \frac{a_n}{b_n}$, então:
\begin{itemize}
\item Se $L = 0$ ou $L = +\infty$, aplica-se o primeiro critério de comparação.
\item Se $0 < L < +\infty$, então as duas séries são da mesma natureza, i.e. ambas convergentes ou ambas divergentes.
\end{itemize}

\chapter{Taylor}
\section{Fórmula de Taylor}
Se $f: A \rightarrow \mathbb{R}$ é uma função $n$ vezes diferenciável em $a \in A$, então existe um único polinómio $T_n(x)$, de grau não superior a $n$, tal que:
$$
f(x) = T_n(x) + o((x-a)^n)
$$
Sendo que $T_n(x)$ é o polinómio de $f$, de ordem $n$, em $a$:
$$
T_n(x) = f(a) + f'(a)(x-a) + \frac{f''(a)}{2!}(x-a)^2 + ... + \frac{f^{(a)}(a)}{n!}(x-a)^n
$$
\section{Fórmula de MacLaurin}
No caso de $a = 0$ dá-se o nome de fórmula de MacLaurin:
$$
f(x) = f(0) + f'(0)x + \frac{f''(0)}{2!}x^2 + ... + \frac{f^{(a)}(0)}{n!}x^n + o(x^n)
$$
\chapter{Primitiva}
\section{Primitivas Imediatas}
\begin{itemize}
\item $\int_{}^{} e^u \cdot u'\ = e^u + C$
\item $\int_{}^{} ku^{k-1} \cdot u'\ = u^k + C$
\item $\int_{}^{} \frac{1}{u} \cdot u'\ = u^k + C$
\item $\int_{}^{} \cos(u) \cdot u'\ = \sin(u) + C$
\item $\int_{}^{} -\sin(u) \cdot u'\ = \cos(u) + C$
\item $\int_{}^{} \frac{1}{1+u^2} \cdot u'\ = \arctan(u) + C$
\end{itemize}

\section{Operações elementares}
\begin{itemize}
\item $\int_{}^{} (f + g)\ = \int_{}^{} f\ + \int_{}^{} g$
\item $\int_{}^{} (kf) = k\int_{}^{} f$
\end{itemize}

\section{Técnicas de primitivação}
\subsection{Primitivação por partes}
$$
\int_{}^{} f'g = fg - \int_{}^{} fg'
$$
\subsection{Primitivação por substituição}
$$
\int_{}^{} f(x) dx = \int_{}^{} f(x(t)) \cdot x'(t) dt|_{t=t(x)}
$$

\section{Funções racionais}
Uma $função$ $racional$ é uma função que é expressa como quociente de duas funções:
$$
f(x) = \frac{P(x)}{Q(x)}
$$
A função diz-se própria se o grau de $P$ é menor que o grau de $Q$.
\subsection{Funções racionais impróprias}
No caso de termos uma função racional imprópria, o primeiro passo consiste
em efetuar a divisão de polinómios, para ficarmos com uma função racional
própria.

\subsection{Funções racionais próprias}
No caso de termos uma função racional própria $f(x) = \frac{P(x)}{Q(x)}$:
\begin{itemize}
\item Decompomos o denominador em fatores
\item Escrevemos a fração racional como soma de frações racionais mais simples
\item Primitivamos
\end{itemize}

\chapter{Integral}
\section{Média integral}
A média integral de $f$ em $[a, b]$ é:
$$
\frac{1}{b-a} \int_{a}^{b}f(t) dt
$$

\section{Função integral indefinido}
Dado $a \in I$, chamamos $integral$ $indefinido$ com origem em $a$  á função $F: I \rightarrow \mathbb{R}$ definida por:
$$
F(x) = \int_{a}^{x}f(t) dt
$$
\subsection{Derivada da função integral indefinido}
Seja $a$ uma constante, $u(x)$ uma expressão de $x$ e $t$ uma variável. Temos
\begin{align}
G(x) = \int_{a}^{u(x)} f(t) dt\\
G'(x) = f(u(x)) \cdot u'(x)
\end{align}
Caso o integral tenha limites não constantes temos:
$$
\left( \int_{\phi(x)}^{\psi(x)} f(t) dt \right)' = f(\psi(x)) \cdot \psi '(x) - f(\phi(x)) \cdot \phi '(x)
$$

\section{Fórmula de Barrow}
Se $f$ é contínua em $I, a, b \in I$ e $G$ é uma qualquer primitiva de $f$, então:
$$
\int_{a}^{b} f(t) dt = G(b) - G(a) = [G(x)]^{b}_a
$$

\section{Métodos de Integração}
\subsection{Integração por partes}
$$
\int_{a}^{b} f'(t)g(t) dt = [f(t)g(t)]^{b}_a - \int_{a}^{b} f(t)g'(t) dt
$$
\subsection{Integração por substituição}
Sendo $x = x(t); dx = \frac{dx}{dt}dt; x(\alpha) = a; x(\beta) = b$
$$
\int_{a}^{b} f(x) dx = \int_{\alpha}^{\beta} f(x(t)) \frac{dx}{dt} dt
$$

\section{Sólidos}
É possível calcular o volume de sólidos através de integração tendo em conta a sua construção, i.e. o facto de que são compostos por infinitas secções de área definida.
\subsection{Sólidos de revolução}
Um sólido de revolução tem secções paralelas todas circulares e pode ser gerado pela rotação de uma figura plana em torno de um eixo.\\
Para calcular o volume destes sólidos basta saber como calcular a área de cada secção circular e integrar essa função do inicio ao fim do objeto em $x$. Por exemplo:
\begin{figure}[h]
\includegraphics[scale=0.33]{Sem Título.png}
\end{figure}
\\
O mesmo pode acontecer em torno de $y$:
\begin{figure}[h]
\includegraphics[scale=0.35]{Sem Título1.png}
\end{figure}
\end{document}